\epigraph{``The Press, Watson, is a most valuable institution, if you only know how to use it."}{--- \textup{Sherlock Holmes}, The Adventure of the Six Napoleons\\[0.2cm] \textup{Sir Arthur Conan Doyle}}

\section*{Aims}
The aim of this project is the development of a tool which generates interactive metro maps of data from RSS feeds, with individual news articles transformed into stations and common themes transformed into \textit{metro lines}. My goal is to reduce the information overload experienced by news consumers, through the provision of contextual links and topic background within the visualisation of the feeds.

The resultant system is a news feed aggregator with graphically structured output, and to the best of my knowledge is first of its kind.

\section*{Outline}

The structure of this dissertation is as follows:
\begin{description}[leftmargin=6em,style=nextline]
	\item [Chapter 1] provides an overview of the background literature upon which this project relies, including a detailed exploration of alternative visualisation formats.
	\item [Chapter 2] describes the scoping of the system, the informal requirements gathering process undertaken, and the rationale behind the significant design decisions.
	\item [Chapter 3] outlines the architecture of the system and describes the algorithms and techniques chosen to transform the data from feed to visualisation.
	\item [Chapter 4] discusses how the methods previously described were implemented in the system, and technical challenges which arose during development.
	\item [Chapter 5] provides a set of example results generated by the system and discusses the effect of altering various parameters on the data.
	\item [Chapter 6] describes the process by which the system was evaluated.
	\item [Chapter 7] summarises both the contributions and limitations of this work, and provides a discussion on future research directions.
\end{description}

