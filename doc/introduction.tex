\vspace{-1cm}
\epigraph{``The Press, Watson, is a most valuable institution, if you only know how to use it."}{--- \textup{Sherlock Holmes}, The Adventure of the Six Napoleons\\[0.2cm] \textup{Sir Arthur Conan Doyle}}

\section*{Aims}
The aim of this project is the development of a tool which generates interactive metro maps of news from RSS feeds, with individual articles transformed into stations and common themes transformed into \textit{metro lines}. Our goal is to reduce the information overload experienced by news consumers by providing contextual links between articles and topics on the map.

The resultant system is a news feed aggregator with graphically structured output, and to the best of our knowledge is first of its kind.

\section*{Outline}
The structure of this dissertation is as follows:
\begin{description}[leftmargin=5.6em,style=nextline]
	\item [Chapter \ref{c:litreview}] provides an overview of the background literature upon which this project relies, including an introduction to information overload and sensemaking, and a justification of appropriate visualisations for news corpora.
	\item [Chapter \ref{c:reqs}] describes the scoping of the system, the informal requirements gathering process undertaken, and the rationale behind the significant design decisions made.
	\item [Chapter \ref{c:implementation}] describes the algorithms and techniques chosen to transform the data from feed to visualisation and discusses how they were implemented in the system along with the trade-offs which were encountered.
	\item [Chapter \ref{c:results}] provides a set of example results generated by the system from different RSS feeds and analyses them from the perspectives of content, layout, and context.
	\item [Chapter \ref{c:evaluation}] describes the process by which we evaluated the system in a user study and discusses the results of our experiments.
	\item [Chapter \ref{c:conclusions}] summarises both the contributions and limitations of this work, and provides a discussion on future research directions.
\end{description}

