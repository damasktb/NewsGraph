\begin{center}
\parbox{\textwidth-3.8cm}{
News overload has emerged as a growing problem in our increasingly connected digital information era. With complex long-running stories unfolding over weeks and months, young adults in particular are left overwhelmed and demotivated, which leads to their disengagement from politics and current events news.\vspace{0.2cm}

This dissertation presents a method for the automatic generation of metro maps based on news content obtained from user-specified RSS feeds. Metro maps are familiar to most adults, and they are intuitive visual metaphors for representing concepts which branch and diverge, such as news stories. The method described performs entity disambiguation and various other NLP techniques to extract a set of topics (\textit{metro lines}) from a news corpus which provide a cohesive summary of its content.\vspace{0.2cm}

The difficulty of drawing unoccluded octilinear metro maps is a barrier to their current utility in InfoVis. Therefore, this dissertation also introduces a heuristic force-directed approach for drawing metro maps, which is refined using multicriteria optimisations taken from neighbouring literature in information cartography.\vspace{0.2cm}

The resultant system is demonstrated using the RSS feeds published by several popular British newspapers, and empirically evaluated in a user study. The results of the study support the hypothesis that metro map users demonstrate greater topic recall than users of an equivalent RSS reader. Lastly, areas for future research are discussed, followed by recommendations for the commercial development of this and similar systems.
}
\end{center}