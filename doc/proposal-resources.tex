I plan to develop the project in Python 2.7, which is freely available, and both maintained and supported by a large community of developers. My decision was motivated by my previous development experience in Python, and the availability of the open source Natural Language Toolkit (NLTK) package\footnote{http://www.nltk.org}, which can perform linguistic processes including tokenisation, lemmatisation and stop word filtering. For other modular tasks such as parsing the RSS feeds, there are a variety of open-source Python packages available.

In terms of graph visualisation, I have two options which, during requirements specification I will have to evaluate and decide between. The first is using an open source imagine library for Python such as PIL\footnote{http://www.pythonware.com/products/pil/} with a graphing class to support the underlying collection model. The second option is for the system to interactive web reports containing the graph structures in JSON/GraphML, with visualisations written in JavaScript.

Any APIs I may use for the system will be free and accessible without delay or constraints for academic use. Currently, I only anticipate using Google's Knowledge Graph API, for which I already have an API key.

Other resources:
\begin{itemize}
	\item For source control and as a backup solution, I have versioned both my source code and documentation files under Git, and hosted them privately on GitHub.
	\item As there is no physical component to the project, there are no additional hardware resources to consider beyond my own computational resources.
	\item For user testing and evaluation, I have classmates in Computer Science and other degree disciplines who have agreed to participate in these activities.
	\item For general guidance, I have arranged weekly supervision meetings to ensure the project progresses at a consistent rate.
\end{itemize}