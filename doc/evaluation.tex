This chapter describes the process by which we designed and conducted a user study to evaluate our metro maps against unstructured RSS feed readers. Results and implications of this study are discussed, with full data available in Appendix \ref{sec:evalresults}. The two metro maps used for the study have been included in Appendix  \ref{sec:evalmaps}.

\section{Scoping the Evaluation}

Typically, information retrieval tools can be evaluated in terms of  accuracy and other quantifiable metrics against a canonical labelled dataset. Our system is less amenable to such evaluation due to the lack of existing datasets. The extensive user study conducted by \cite{GeneratingInformationMaps} evaluated four distinct aspects of their metro map system; (1) accuracy of the algorithm's document selection in respect to a query; (2) user fact retrieval time; (3) user understanding at a macro level; and (4) task performance of users with a Metro Map versus an unstructured list.

For our system, the first aspect (accuracy of retrieval for queries) is not of interest, because our maps are not query-based or selected from a large fixed corpus with a reasonable likelihood of Type II errors. The second and third, while relevant to our system, do not effectively capture the spirit of its intended purpose. Our metro maps are not designed for search tasks; if a user has a specific question about an article in a metro map, they will most likely still have to open and read the article itself for the answer. The maps the generated by the system exist to help the user form high-level connections between topics and to make sense of those they do not understand. A search task would have to ask artificially simple questions in order for the answers to be derivable using our maps alone, and would therefore not be representative of typical news consumption tasks.

The final aspect is therefore the focus of this evaluation. The goal of the experiments is to determine whether the structure of the metro maps alone provides a better overview to news content than a the traditional RSS feed view. We answer this question using \possessivecite{TowardsAnOptimalResolutionToInformationOverload} dimensions of information overload to choose two perspectives to evaluate:

\begin{itemize}
	\item Users' perceptions of \textbf{information quantity}. \par
		Since information overload is based partially on readers' intrinsic estimations of information quantity, we examine the effect that varying information format has on these estimations.
	\item The \textbf{contextual quantity} of our maps, as measured by task performance. \par
		We conduct an experiment similar to that of \cite{scattergather}, measuring the effect of varying information format on the breadth and depth of users' topic recollection.
\end{itemize}

\section{Experimental Hypotheses}

Considering perception and performance as our two separate measures of information overload, we developed the following hypotheses to be tested in two corresponding experiments:

\begin{hyp}[\ref{hyp:eq:estimation}]
\label{hyp:estimation}
Users' estimates of the number of articles in a metro map are lower than their estimates of the same number of articles when displayed in a list form.
\end{hyp}
\vspace{-0.6cm}
\begin{align*}
	H_{0}(\ref{hyp:estimation}) &: \mu{E_{map}} = \mu{E_{list}} \\
	H_{1}(\ref{hyp:estimation}) &: \mu{E_{map}} < \mu{E_{list}} \numberthis
	\label{hyp:eq:estimation}
\end{align*}

\begin{hyp}[\ref{hyp:eq:index}]
\label{hyp:index}
Users recall a higher number of topics after using a metro map representation than after using the RSS feed list view.
\end{hyp}
\vspace{-0.6cm}
\begin{align*}
	H_{0}(\ref{hyp:index}) &: \mu{T_{map}} = \mu{T_{list}} \\
	H_{1}(\ref{hyp:index}) &: \mu{T_{map}} > \mu{T_{list}} \numberthis
	\label{hyp:eq:index}
\end{align*}

Where $\mu{E}_{map}$ and $\mu{E}_{list}$ are the mean estimate values for the number of articles contained within a map and list respectively, and $\mu{T}_{map}$ and $\mu{T}_{list}$ are the mean number of topics successfully recalled.

\section{Experimental Design}

In order to increase the number of subjects participating in each condition and control for individual variances in ability, both experiments were conducted within-groups. With each subject participating in both conditions (\textit{map} and \textit{list}), two corpora of news articles were required. Subjects were therefore assigned to one of four groups, with counterbalancing of both the order in which the conditions were conducted, and the order in which the two corpora were read (Table \ref{tab:experimentalgroups}.) Each group contained four participants.\\

\begin{table}[htbp!]
\centering
\begin{tabular}{|r|c|c|}
\hline
  & A, B & B, A\\
\hline
\textit{List}, \textit{Map} & (Group 1) \textit{List} A \textbf{then} \textit{Map} B & (Group 3) \textit{List} B \textbf{then} \textit{Map} A \\
\hline
\textit{Map}, \textit{List} & (Group 2) \textit{Map} A \textbf{then} \textit{List} B & (Group 4) \textit{Map} B \textbf{then} \textit{List} A \\
\hline
\end{tabular}
\caption{Participant groups} \label{tab:experimentalgroups}
\end{table}

\subsubsection{Participant Criteria}
In total, sixteen participants (four female, eight male, mean age = 21.8 years) enrolled in the study. All had perfect or corrected-to-perfect vision and hearing, and were undergraduates at the University of Bath.

No participants were excluded on the basis of how much or how little news they consumed on a daily basis. Similarly, although it was recorded whether participants had previously used or currently use an RSS feed reader, this was not used as a selection criterion. 

\section{Methodology}

\subsection{Estimate Task} \label{task:est}

The first experiment was designed to test Hypothesis \ref{hyp:estimation}. Participants were exposed to both modalities for 60 seconds (order dependent on their assigned group), after which they estimated the number of articles represented in each. To prevent conscious or subconscious counting by participants during the second condition, subject were exposed to both datasets, one immediately after another. Only after exposure to the second set were they were asked to estimate the number of articles in each.

\subsection{Index Task}

The second experiment, which was designed to test Hypothesis \ref{hyp:index}, placed significantly more cognitive load on the participants. The task is the same as \possessivecite{scattergather} evaluation of the Scatter/Gather Browser, which required participants to draw a topic index, or \textit{tree}, representing their understanding of topic structure within a corpus.

For each of the two conditions, our participants first spent three minutes reading and exploring the news represented.\footnote{As this task was completed after the Estimate Task (\ref{task:est}), the total time spent interacting with the data was actually four minutes.} After the three minutes, the modality was minimised and the participants were given two minutes to write a topic index based on the news they had just read. An example topic index based on news in a different domain was shown before this task began to ensure participants knew what they were supposed to produce. Both datasets (and therefore all four conditions) contained 31 articles.

Finally, participants were asked to complete a Self-Assessment Manikin \citep{measuringemotion} to rate the pleasure (labelled from `unhappy' to 'happy'), arousal (`calm' to `excited') and dominance (`not in control' to `in control') they felt as a result of completing the index task. This process was then repeated with the second modality and data set, according to the participant's grouping.

\subsection{Variables}

The following section describes the independent variable in both studies, the dependent variables we measured, and the control variables which we either controlled through the design of the study or analysed as potential covariates.

\subsubsection{Independent Variables}

The independent variable in both experiments was the display modality; a nominal value which was either \textit{metro map} or \textit{list}. Participants completed the same tasks with both a metro map and a list modality, using a different news corpus for each.

\subsubsection{Dependent Variables}

\begin{itemize}
	\item\textbf{Article Estimate (Hypothesis \ref{hyp:estimation})} \par
		This is the raw estimate of the number of articles, and is measured on a ratio scale. 
	\item\textbf{Primary topics (Hypothesis \ref{hyp:index})} \par
		This is the number of `top-level' topics identified in a participant's index which are present in the corpus, measured on a ratio scale. If participants identified a single primary topic (e.g. `America'/`The USA'), this top level of the index was ignored and primary topics were counted as those in the next level.
	\item\textbf{Topic index depth (Hypothesis \ref{hyp:index})} \par
		This is the maximum depth reached in a participant's index, measured on a ratio scale.
	\item\textbf{Total topics (Hypothesis \ref{hyp:index})} \par
		This is the total number of topics identified at all levels in a participant's index which are present in the corpus, measured on a ratio scale.
	\item\textbf{Self-Assessment Manikin (SAM) score (Hypothesis \ref{hyp:index})} \par
		The valence, arousal and dominance experienced by participants after completing the Index Task. These were measured on a 9-point scale, as shown below.

\end{itemize}

\subsubsection{Control Variables}
There were a number of participant characteristics and other covariates which could confound the results of one or both experiments. The majority of these are controlled for due to all subjects participating in both conditions, or due to our participant selection process, but for completeness they are all listed below.

\begin{itemize}
	\item\textbf{Level of formal education} \par
		It was expected that differing levels of formal education would impact the relative performance of participants, however all participants were undertaking the final year of an undergraduate degree, so we did not account for this as a covariate.
	\item\textbf{Weekly news consumption time} \par
		It was expected that participants who consume more news on average would perform uniformly better across both conditions, but as subjects participated in both conditions, we do not expect to need to account for this as a covariate. This variable was measured on a discrete scale of hours per week.
	\item\textbf{Past or present use of an RSS feed reader} \par
		It was expected that participants who used or had previously used an RSS reeder may perform disproportionally better with the \textit{list} condition than with the \textit{map}. This variable was measured on an ordinal scale of ``Current Usage", ``Previous usage", or ``No usage." This factor was examined as a possible covariate.
	\item\textbf{Topic Familiarity} \par
		Since all subjects participated in both conditions, each subject completed the \textit{metro map} condition with one set of articles and the \textit{list} condition with the other set. The two corpora (A and B) were extracted from the same RSS Feed\footnote{The Guardian, US News, week commencing 20/03/2017.} within the same week, to ensure background knowledge wouldn't confound performance between A and B.
\end{itemize}

\section{Results and Discussion}

\subsection{Testing Hypothesis \ref{hyp:estimation}: Estimate Task}

Our first hypothesis predicted that users' estimates for the number of articles represented on a map would be lower than their estimates of the same number of articles shown in the form of a list.


\subsection{Testing Hypothesis \ref{hyp:index}: Index Task}

Because both experiments were conducted within-groups and every participant was therefore observed twice, observations are not independent, so we use a paired t-test to test the significance of our findings.

\begin{itemize}
	\item Testing \ref{hyp:estimation} - No significant difference
	\item Testing \ref{hyp:index} - Enough evidence to reject $H_0$, and this is due to more primary topics rather than more depth, as expected.
	\item Significant effect of news consumption on list primary topics
	\item Effect of modality on primary topics which were metro lines
	\item Effect of modality on SAM Score
	\item Correlation of pleasure and dominance within SAM Scores
	\item Negative correlation between arousal for map/list
\end{itemize}

\subsection{Qualitative Analysis}

One participant, during the index task for the list condition, asked whether topics could be repeated, ``Because this one relates to both of those [parent topics]''. He then remarked, ``This really does suit being on a graph, doesn't it.''

Two participants added a category labelled `miscellaneous' or 'other' to their topic indexes, one during the map condition and the other during the list condition, but both on dataset B, suggesting the topics in that set may have been more disparate.

One participant, despite identifying more topics (both primary and total) using a metro map, reported lower happiness and feeling less in control with the map, stating that ``The list was easier. I could link the topics myself in a way that felt more natural. Using the map I didn't always agree with the categorisation."

A different participant whose scores were similar for both conditions said ``It was easier to find stuff I cared about [using the map] with nothing in the way.''

\section{Summary}