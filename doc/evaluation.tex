Typically, information retrieval tools can be evaluated in terms of  accuracy and other quantifiable metrics against a canonical labelled dataset. Our system is less amenable to such evaluation due to the lack of existing datasets. The extensive user study conducted by \cite{GeneratingInformationMaps} evaluated four distinct aspects of their metro map system; (1) Accuracy of the algorithm's document selection in respect to a query; (2) User fact retrieval time; (3) User understanding at a macro level; and (4) task performance of users with a Metro Map versus an unstructured list.

For our system, the first aspect (accuracy of retrieval for queries) is not of interest, because our maps are not query-based or selected from a large fixed corpus with a reasonable likelihood of type II errors.

The second and third, while relevant to our system, do not effectively capture the spirit of its intended purpose. Our metro maps are not designed for search tasks; if a user has a specific question about an article in a metro map, they will most likely still have to open and read the article itself for the answer. The maps exist to help the user form high-level connections between topics and to make sense of those they do not understand. A search task would have to ask artificially simple questions in order for the answers to be derivable using our maps alone, and would therefore not be representative of typical news consumption tasks.

The final aspect is therefore what we chose to focus on. The goal of the experiment is to determine whether the structure of the metro maps alone provides a better overview to news content than a the traditional RSS feed view. We answer this question using \possessivecite{TowardsAnOptimalResolutionToInformationOverload} dimensions of information overload to choose two perspectives to evaluate:

\begin{itemize}
	\item Users' perceptions of \textbf{information quantity}. \par
		Since information overload is based partially on readers' intrinsic estimations of information quantity, we examine the effect that varying information format has on these estimations.
	\item The \textbf{contextual quantity} of our maps, as measured by task performance. \par
		We conduct an experiment similar to that of \cite{scattergather}, measuring the effect of varying information format on the breadth and depth of users' topic recollection.
\end{itemize}

\section{Experimental Hypotheses}

Considering perception and performance as our two separate measures of information overload, we developed the following hypotheses to be tested in two separate experiments:

\begin{hyp}[\ref{hyp:eq:estimation}]
\label{hyp:estimation}
Users' estimates of the number of articles in a metro map are lower than their estimates of the same number of articles when displayed in a list form.
\end{hyp}
\vspace{-0.6cm}
\begin{align*}
	H_{0}(\ref{hyp:estimation}) &: \mu{E_{map}} = \mu{E_{list}} \\
	H_{1}(\ref{hyp:estimation}) &: \mu{E_{map}} < \mu{E_{list}} \numberthis
	\label{hyp:eq:estimation}
\end{align*}

\begin{hyp}[\ref{hyp:eq:index}]
\label{hyp:index}
Users recall a higher number of topics after using a metro map representation than after using the RSS feed list view.
\end{hyp}
\vspace{-0.6cm}
\begin{align*}
	H_{0}(\ref{hyp:index}) &: \mu{T_{map}} = \mu{T_{list}} \\
	H_{1}(\ref{hyp:index}) &: \mu{T_{map}} > \mu{T_{list}} \numberthis
	\label{hyp:eq:index}
\end{align*}

Where $\mu{E}_{map}$ and $\mu{E}_{list}$ are the mean estimate values for the number of articles contained within a map and list respectively, and $\mu{T}_{map}$ and $\mu{T}_{list}$ are the mean number of topics successfully recalled.

Due to the small sample size of the experiment, the null hypothesis was tested at 5\% ($\alpha = 0.05$) significance. 


\section{Experimental Design}

In order to increase the number of subjects participating in each condition and control for individual variances in ability, both experiments were conducted within-groups. With each subject participating in both conditions (\textit{map} and \textit{list}), two corpora of news articles were required. Subjects were therefore assigned to one of four groups, with counterbalancing of the order in which the conditions were conducted (Table \ref{tab:experimentalgroups}.) All four groups contained the same number of participants. \\

\begin{table}[htbp!]
\centering
\begin{tabular}{|r|c|c|}
\hline
  & A, B & B, A\\
\hline
\textit{List}, \textit{Map} & (Group 1) \textit{List} A \textbf{then} \textit{Map} B & (Group 3) \textit{List} B \textbf{then} \textit{Map} A \\
\hline
\textit{Map}, \textit{List} & (Group 2) \textit{Map} A \textbf{then} \textit{List} B & (Group 4) \textit{Map} B \textbf{then} \textit{List} A \\
\hline
\end{tabular}
\caption{Participant Groups} \label{tab:experimentalgroups}
\end{table}

To prevent conscious or subconscious counting by participants during the second condition, subject were exposed to both datasets for 60 seconds, one immediately after another. After exposure to the second set, they were asked to estimate the number of articles in both sets.

\subsection{Planning}





\subsection{Variables}

\subsubsection{Independent Variables}

The independent variable in both experiments is the display modality; a nominal value which is either \textit{metro map} or \textit{list}. Participants will complete the same tasks with both a metro map and a list modality, using a different news corpus for each.

\subsubsection{Dependent Variables}

\begin{itemize}
	\item\textbf{Article Estimate (Hypothesis \ref{hyp:estimation})} \par
		This is the estimate of the number of articles divided by the actual number of articles present, and is measured on a ratio scale, with a perfect estimate being 1.0. 
	\item\textbf{Primary topics (Hypothesis \ref{hyp:index})} \par
		This is the number of `top-level' topics identified in a participant's index which are present in the corpus, measured on a ratio scale.
	\item\textbf{Topic index depth (Hypothesis \ref{hyp:index})} \par
		This is the maximum depth reached in a participant's index, measured on a ratio scale.
	\item\textbf{Total topics (Hypothesis \ref{hyp:index})} \par
		This is the total number of topics identified at all levels in a participant's index which are present in the corpus, measured on a ratio scale.
	\item\textbf{Self-Assessment Manikin (SAM) score (Hypothesis \ref{hyp:index})} \par
		The valence, arousal and dominance experienced by participants after completing Experiment 2.

\end{itemize}

\subsubsection{Control Variables}
There were a number of participant characteristics and other covariates which could confound the results of one or both experiments. The majority of these are controlled for due to all subjects participating in both conditions, or due to our participant selection process, but they are all listed below.

\begin{itemize}
	\item\textbf{Level of formal education} \par
		It was expected that participants' respective levels of formal education may impact their performance, however, as the experiments are within-groups and all of our participants were undertaking the final year of an undergraduate degree, we did not account for this as a covariate.
	\item\textbf{Weekly news consumption time} \par
		It was expected that participants who consume more news on average will perform uniformly better across both conditions, but as subjects participate in both conditions we do not expect to need to account for this as a covariate. This was measured on a discrete scale of hours per week.
	\item\textbf{Past or present use of an RSS feed reader} \par
		As above, it was expected that participants who currently use or have previously used an RSS reeder may perform disproportionally better with the \textit{list} condition than with the \textit{map}. This variable was measured on an ordinal scale of ``Current Usage", ``Previous usage", or ``No usage." This factor was examined as a possible covariate.
	\item\textbf{Topic Familiarity} \par
		Since all subjects participated in both conditions, each subject completed the \textit{metro map} condition with one set of articles and the \textit{list} condition with the other set. The two corpora (A and B) were extracted from the same RSS Feed\footnote{The Guardian, US News, week commencing 20/03/2017.} within the same week, to ensure background knowledge wouldn't confound performance between A and B.
	\item\textbf{Relative complexities of the two corpora} \par
	Although both A and B contained exactly the same number of articles (31), corpus A was significantly more connected, with 39 links; 18\% more edges than corpus B's 33. This factor is discussed more later in the chapter.
\end{itemize}


\subsubsection{Participant Criteria}
All participants recruited for the study were undergraduates between the ages of 20 and 23. They all had perfect or corrected-to-perfect vision and hearing. There were no gender requirements, but the majority of subjects were male.

No participants were excluded on the basis of how much or how little news they consumed on a daily basis. Similarly, although it was recorded whether participants had previously used or currently use an RSS feed reader, this was not used as a selection criterion. 


\section{Methodology}

\section{Results}

Because both experiments were conducted within-groups and every participant was therefore observed twice, observations are not independent, so we use a paired t-test to test the significance of our findings.

\section{Summary}